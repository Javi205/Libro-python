\subsection{Apertura, lectura y escritura de archivos}
Para trabajar con archivos en Python, se utiliza la función open(). Esta función toma dos argumentos: el nombre del archivo y el modo de apertura.\\

El modo de apertura determina cómo se puede acceder al archivo. Los modos de apertura más comunes son:
\begin{itemize}
    \item r: Lectura. El archivo se abre en modo de lectura.
    \item w: Escritura. El archivo se abre en modo de escritura. Si el archivo no existe, se crea uno nuevo. Si el archivo existe, se sobrescribe su contenido.
    \item a: Adición. El archivo se abre en modo de adición. Los datos se escriben al final del archivo.
\end{itemize}

Por ejemplo, para abrir un archivo de texto en modo de lectura, se puede usar el siguiente código:

Una vez que un archivo está abierto, se puede leer su contenido usando la función read(). Esta función devuelve una cadena con todo el contenido del archivo.\\

Por ejemplo, para leer el contenido de un archivo de texto, se puede usar el siguiente código:

También se puede leer el contenido de un archivo de texto línea por línea usando un bucle for:

Para escribir datos en un archivo, se usa la función write(). Esta función toma una cadena como argumento y la escribe al final del archivo.\\

Por ejemplo, para escribir el siguiente texto en un archivo de texto:

Se puede usar el siguiente código:

\subsection{Operaciones comunes de archivos}
Además de la apertura, lectura y escritura, Python ofrece otras operaciones comunes de archivos.

\begin{itemize}
    \item seek(): Permite mover el puntero de lectura o escritura a una posición determinada en el archivo.
    \item tell(): Devuelve la posición actual del puntero de lectura o escritura en el archivo.
    \item flush(): Escribe cualquier dato que aún no se haya escrito en el archivo al disco.
    \item close(): Cierra el archivo y libera los recursos asociados con él.
\end{itemize}

\subsection{Manejo de archivos con el bloque ``with''}
El bloque with es una forma segura y conveniente de abrir y cerrar archivos. Cuando se usa el bloque with, el archivo se abre automáticamente al principio del bloque y se cierra automáticamente al final del bloque.\\

Por ejemplo, el siguiente código abre un archivo de texto en modo de lectura y luego imprime su contenido:

Por ejemplo, el siguiente código abre un archivo de texto en modo de lectura y luego imprime su contenido:

El uso del bloque with evita errores comunes, como olvidar cerrar un archivo abierto.

