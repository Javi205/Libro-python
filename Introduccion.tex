\subsection{¿Qué es Python?}
Python es un lenguaje de programación, lo que este está diseñado para una fácil comprensión para los seres humanos. Fue creado por Guido van Rossum, lanzó su primera versión en 1991, Python ha ganado popularidad y se ha convertido en uno de los lenguajes de programación.\\

El diseño de Python se centra en la simplicidad y la elegancia lo que facilita a los desarrolladores escribir códigos claros y concisos.

Python es un lenguaje interpretado, lo que significa que el código escrito por los programadores se traduce a un lenguaje intermedio que luego es ejecutado por el intérprete de Python, esto permite un rápido desarrollo, ya que los programadores pueden ver el resultado de su código inmediatamente después de escribirlo.\\

Utiliza ``indentación'' para definir bloques de código, lo que elimina la necesidad de palabras claves adicionales, esto hace que el código sea limpio y fácil de leer.
Python es un lenguaje multiparadigma, lo que significa que está orientada a programación con objetos como programación estructurada, esto permite a los desarrolladores utilizar diferentes enfoques según las necesidades de su proyecto. Además Python tiene una gran biblioteca estándar que proporciona módulos y paquetes para diversas aplicaciones, desde desarrollo web hasta procesamiento de datos
y cálculos científicos 

\subsection{Ventajas de usar Python}
\begin{enumerate}
    \item Es un lenguaje de sintaxis amplia y legible: Al ser lenguaje de programación de alto nivel, diseñado para que los algoritmos sean expresados de forma clara y fácilmente entendibles por los seres humanos.
    \item Ampliamente utilizado en múltiples campos: Su amplia variedad de usos lo ha dejado como el primero en el top 10 de los lenguajes de programación más utilizados según Tiobe, extraídos de las habilidades más desarrolladas por desarrolladores, empresas del sector y terceros.
    \item Python posee una gran cantidad de bibliotecas y frameworks : La gran cantidad de usos de Python se traduce en múltiples librerías y frameworks que ayudan a llevar a cabo funcionalidades. En sí mismo ya tiene una biblioteca estándar y podemos encontrar hasta 135. 000+ más para diversas aplicaciones. Sin embargo, entre las más populares según el sitio de AWS se encuentran Matplotlib, Pandas, Request, Numpy, Keras y OpenCV-Python. Esta variedad no se limita solo a las librerías. Así mismo la podemos encontrar en los marcos o frameworks, que facilitan el proceso de creación debido a que ahorra el proceso de escritura de un código. 
    \item Fácil portabilidad: Python es uno de los lenguajes de programación más portátiles y versátiles disponibles. Debido a que es un lenguaje de programación interpretado, en lugar de un lenguaje compilado, se puede ejecutar en una amplia variedad de sistemas operativos y plataformas de hardware sin necesidad de realizar ajustes o cambios significativos en el código fuente.
    \item Tiene una gran comunidad de desarrolladores : Es una herramienta que constantemente evoluciona para suplir las necesidades que poco a poco van surgiendo en el campo de la tecnología, como hemos visto hasta ahora en sus usos para el machine learning.
    \item Multiplataforma: Python es uno de esos lenguajes de programación que puede ser ejecutado en cualquier sistema operativo en el cual se opere. Así es: no importa si se trata de Windows, Linux, macOS, y otros, este se puede ejecutar sin problema. Y, lo mejor, es que se desarrolla el código una única vez y podrá emplearse en los demás SO. 
\end{enumerate}