
Es necesario saber diferenciar las variables y tipos de datos al momento de programar

\subsection{Tipos numéricos}
Los tipos de datos numéricos son usados en programación para hacer cálculos y manipular números, sus tipos son:
\begin{itemize}
\item Enteros(int): estos representan los números enteros no decimales por ejemplo: -1, 0, 69, 13, -7
\item Punto Flotante (float):Representan números con decimales.Ejemplo: 3.14, -0.5, 2.718, 100.0.
\end{itemize}
Para ejecutar estos tipos de datos numéricos basta con asignarle un valor.

Ejemplo:

\subsection{Tipos de texto}
Los tipos de variables de texto son fundamentales en programación para manejar y manipular cadenas de caracteres.\\

En las variables de texto tenemos el string o ``texto de verdad'', el cual puede contener desde una palabra a un párrafo completo, ademas en una variable de tipo string podemos cargar una cantidad variable de caracteres. Estos caracteres una vez cargados a un string no se pueden cambiar

Ejemplo:

\subsection{Tipos de secuencia}
Las secuencias son estructuras de datos fundamentales que permiten almacenar colecciones ordenadas de elementos.

Una son las listas las cuales son secuencias mutables de elementos.Los elementos pueden ser de diferentes tipos, incluso otras listas.

Ejemplo:

\begin{itemize}
    \item Agregar Elementos: lista.append(elemento)
    \item Acceso por Índice: lista[indice]
    \item Slicing: lista[inicio:fin]
    \item Modificar Elementos: lista[indice] = nuevo\_valor
    \item Eliminar Elementos: del lista[indice]
\end{itemize}

La otra secuencia son las tuplas las cuales son secuencias de elementos ordenadas e inmutables y que se utilizan para datos que por su naturaleza, no deben cambiar.

Una tupla puede ser creada poniendo los valores separados por comas y entre paréntesis. Por ejemplo, podemos crear una tupla que tenga el nombre y el apellido de una persona de la siguiente manera:

\subsection{Tipos de mapeo}
Uno de los tipos de mapeo es el Diccionario, a veces llamado Mapa en algunos lenguajes el cual es:

Una estructura para programación general que contiene un número dinámico de entradas, donde cada entrada tiene una clave única y un valor asociado.
Se puede agregar una nueva entrada, eliminar y cambiar su valor. Y usualmente son escritos entre corchetes.\\

Ejemplo:

\subsection{Tipos booleanos}

Un operador de tipo booleano es un dato que solo puede tener dos valores ya que representa valores de lógica binaria, y por lo general se pueden mostrar con un dato que sea Verdadero o Falso.
