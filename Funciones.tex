
Las funciones son bloques de código reutilizables que realizan una tarea específica. Permiten dividir un programa en partes más pequeñas y manejables, lo que facilita la comprensión y el mantenimiento del código. La modularidad se refiere a la práctica de dividir un programa en módulos o funciones independientes que pueden ser desarrolladas y probadas de forma separada. Esto promueve el código limpio y organizado, facilitando la colaboración en equipos de desarrollo.

\subsection{Definición y llamada de funciones}

Definición de Funciones:\\
  
En la mayoría de los lenguajes de programación, las funciones se definen con la palabra clave ``def'' (en Python), ``function'' (en JavaScript), ``fun'' (en Kotlin), o ``void'' (en C++), seguido del nombre de la función y una lista de parámetros entre paréntesis. Por ejemplo, en Python:

Llamada de Funciones:\\

Para utilizar una función, se realiza una llamada a la función, pasando los valores necesarios como argumentos. Los argumentos son los valores reales que se pasan a la función durante la llamada. Por ejemplo:

En este caso, ``Juan'' es el argumento que se pasa a la función ``saludar''.

\subsection{Argumentos y parámetros}
Parámetros de Funciones:\\

Los parámetros son variables que se utilizan en la definición de la función para aceptar valores. En el ejemplo anterior, ``nombre'' es un parámetro de la función ``saludar''.

Argumentos de Funciones:\\

Los argumentos son los valores reales que se pasan a la función durante su llamada. En el ejemplo de la llamada a la función `saludar("Juan")`, "Juan" es el argumento que se pasa al parámetro ``nombre'' de la función ``saludar''.

Las funciones pueden tener múltiples parámetros y se pueden pasar argumentos de diferentes tipos (números, cadenas, listas, etc.). Por ejemplo:

\subsection{Ámbito de variables}
\subsection{Módulos y su importación}